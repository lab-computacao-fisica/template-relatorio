\section{Introdução}

Texto de introdução do relatório técnico. \\

Esse texto deve apresentar os aspectos principais a serem abordados no relatório, bem como uma síntese da metodologia, se for o caso, e do contexto geral do projeto. \\

Também convém inserir uma breve explicação de como o relatório está estruturado, suas partes e capítulos. \\

Início da parte textual do trabalho. Tem como finalidade dar ao leitor uma visão concisa do tema investigado, ressaltando se: o assunto de forma delimitada, ou seja, enquadrando-o sob a perspectiva de uma área do conhecimento, de forma que fique evidente sobre o que se está investigando; a justificativa da escolha do tema; os objetivos do trabalho; o objeto de pesquisa que será investigado durante o transcorrer da pesquisa. \\

Todo texto deve ser digitado em fonte Times New Roman ou Arial, tamanho 12, inclusive a capa, com exceção das citações com mais de três linhas, notas de rodapé, paginação, dados internacionais de catalogação-na-publicação (ficha catalográfica), legendas e fontes das ilustrações e das tabelas, que devem ser em fonte tamanho 10. O texto deve ser justificado, exceto as referências, no final do trabalho, que devem ser alinhadas a esquerda. \\

Todos os autores citados devem ter a referencia incluída em lista no final no trabalho \\